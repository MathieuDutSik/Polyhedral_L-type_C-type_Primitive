\documentclass[12pt]{article}
\usepackage{amsfonts, amsmath, latexsym, epsfig}
\usepackage{epsf}
\usepackage{url}
\title{Automorphism group of Lorentzian lattices}
\DeclareMathOperator{\Stab}{Stab}
\DeclareMathOperator{\rank}{rank}

\def\QuotS#1#2{\leavevmode\kern-.0em\raise.2ex\hbox{$#1$}\kern-.1em/\kern-.1em\lower.25ex\hbox{$#2$}}


%\usepackage{vmargin}
%\setpapersize{custom}{21cm}{29.7cm}
%\setmarginsrb{1.7cm}{1cm}{1.7cm}{3.5cm}{0pt}{0pt}{0pt}{0pt}
%marge gauche, marge haut, marge droite, marge bas.

\begin{document}
\newcommand{\RR}{\ensuremath{\mathbb{R}}}
\newcommand{\NN}{\ensuremath{\mathbb{N}}}
\newcommand{\QQ}{\ensuremath{\mathbb{Q}}}
\newcommand{\CC}{\ensuremath{\mathbb{C}}}
\newcommand{\ZZ}{\ensuremath{\mathbb{Z}}}
\newcommand{\TT}{\ensuremath{\mathbb{T}}}
\newtheorem{proposition}{Proposition}
\newtheorem{theorem}{Theorem}
\newtheorem{corollary}{Corollary}
\newtheorem{lemma}{Lemma}
\newtheorem{problem}{Problem}
\newtheorem{conjecture}{Conjecture}
\newtheorem{claim}{Claim}
\newtheorem{remark}{Remark}
\newtheorem{definition}{Definition}
\newcommand{\qed}{\hfill $\Box$ }
\newcommand{\proof}{\noindent{\bf Proof.}\ \ }

\begin{center}
\begin{tabular}{|c|c|c||c|c|c|}
\hline
n & nr. sec. c. & nr. cont. c. & n & nr. sec. c. & nr. cont. c.\\
\hline
1 & 7 & 7 & 9 & 21132 & 33085\\
2 & 37 & 39 & 10 & 22221 & 37601\\
3 & 146 & 161 & 11 & 18033 & 32821\\
4 & 535 & 613 & 12 & 10886 & 21292\\
5 & 1681 & 2021 & 13 & 4713 & 9709\\
6 & 4366 & 5543 & 14 & 1318 & 2787\\
7 & 9255 & 12512 & 15 & 222 & 397\\
8 & 15692 & 22806 &  &  & 
\\
\hline
\end{tabular}
\end{center}
\end{document}

